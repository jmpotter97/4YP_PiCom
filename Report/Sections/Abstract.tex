\documentclass[../main.tex]{subfiles}
	% ABSTRACT
\begin{document}

%\begin{flushleft}
\noindent
Software-defined radio test beds are valuable resources for verifying theoretical work on digital communications, however they are often too expensive to justify putting together. This paper investigates the design of a software-designed radio (SDR) test bed using two Raspberry Pis and a number of commercially available off-the-shelf chips.
The test bed is designed to be significantly lower cost than current SDR platforms, which use expensive custom software radio boards as the radio interface.
The result is a system with a retail price of $\mathsterling165$, which constitutes a significant reduction in price in comparison to any of the other test beds discussed.
The construction and operation of the test bed are investigated for the understanding of the reader.
This architecture is analysed and the issues dealt with such that it may be extended or improved.
The test bed is then tested for baseband transmission of On-Off Keying, 4-level Pulse Amplitude Modulation and 16-Quadrature Amplitude Modulation, however a problem encountered with the digital analogue converter used made it unreliable for Orthodonal Frequency Division Multiplexing.
The limits of the test bed are analysed, and although it performs reasonably well given the low budget, the prototyped design prevents the test bed from working well at high frequencies where exposed wires emit interfering radiation to adjacent wires causing impairments.
Suggestions to mitigate these problems and improve on the architecture are made.\\

%\end{flushleft}

\end{document}